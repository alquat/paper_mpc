
\section{Introduction}

\IEEEPARstart{P}{ower} electronic converters (PECs) are ubiquitous, serving as critical interfaces in electrification by connecting loads, energy storage systems, sources, and the grid. For example, PECs are used to convert alternating current (AC) to direct current (DC) in electric vehicles (EVs), transform DC to AC in photovoltaic systems to align with grid frequency and voltage and facilitate operations in high-power applications and various industrial processes, mainly electric drives. 
As Europe progresses in its green transition and electrification actions, 
it becomes clear how PECs have a major role in orchestrating 
a large network of \textit{prosumers}, i.e., flexible loads that could serve as consumers or producers based on the operational
grids. Today, most industrial applications are still unidirectional, meaning they consume energy according to their strategy without taking grid conditions into account, primarily due to the fact that energy generation was traditionally dominated by large plants operating on a fixed schedule to meet specific demands during designated hours.
However, in addition to the inverter-based grid, and stability challenges,
with the energy landscape gradually shifting from centralized to decentralized and stochastic generation, it also pushes for significant technological advancements. 


However, the aggregation of this flexibilities come with a price,
especially in terms of data aggregation, computational resources, scalability and advances control 
techniques that are able to interact with cloud architecture, PLCs and APIs. (Read paper Musumeci) 


Small summaries of paper analyzing could platform with gree transitions


\cite{bagherzadeh2020integration} analyzes the challanges in communication, storages and computational capabilities in 
massive streams of data, while \cite{shahinzadeh2018green} presents how IoT benefit the accurate forecasting and predictive mantainance ensuring high security levels. 
Paper \cite{hossein2020internet} extensively review the literature on the application of IoT in energy sectors ans smart grids, by 
distinguishes the the transmission and distribution levels, where IoT can be applied to energy efficiency, aggregation of distributed generations and electric vehicles
aggregation (V2G), from the demand side where IoT can be used for battery energy storage management and control to smart building control. 

Wi-Fi, Bluetooth, ZigBee \cite{karunarathne2018wireless} LTE-4G and 5G networks \cite{li20185g}



As an example in paper 1 the authors studies the return of investment of the design
of distributed energy resources. Other studies focuses on how a distributed scenario could benefit the industry, 
or how policy maker could push for emerging markets for distribute resources. On the bottom level, the integration of grid-based inverters
playing a pivotal role in designing a grid able to maintain a power quality and stability.
In fact it is known that with the increasing amount of renewables, the grid is being more exposed to fluctation 
and stochasticty mainly driven by the weather forecast, requiring the energy storage and ancillary service includes 
managing voltage levels and frequency to prevent grid disturbances.
Another part of the literature is focusing on scheduling and resource allocation under uncertainty. 
Methods like stochastic or robust optimization are now becoming more popular in order to minimize Value at Risk (VAr), under uncertain conditions. 





