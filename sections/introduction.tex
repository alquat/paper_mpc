
\section{Introduction}

\IEEEPARstart{T}{his} document is a template for authors. If you are reading a paper or PDF version of this document, please download the electronic file, TRANS-JOUR.zip, from the TIE Web site at \url{http://www.ieee-ies.org/pubs/transactions-on-industrial-electronics} so you can use it to prepare your manuscript.

After unzipping, choose Word or LATEX option. Open ALL\_xx-TIE-xxxx.docx for Word or ALL\_x-TIE-xxxx.tex for LATEX (highly recommended).

These types of manuscript are accepted in TIE:

\section{System Description}

The energy production landscape is gradually shifting from a centralized model towards 
more decentralized systems. 
This transformation is evident in the increased deployment and integration of 
renewable energy sources. Single wind turbine, and rooftop PV have are moving fast towards wind and solar farm.  
 As result of this transition, an increasing numerous works focused on potential issues arising from 
a scattered production at different level. As an example in paper 1 the authors studies the return of investment of the design
of distributed energy resources. Other studies focuses on how a distributed scenario could benefit the industry, 
or how policy maker could push for emerging markets for distribute resources. On the bottom level, the integration of grid-based inverters
playing a pivotal role in designing a grid able to maintain a power quality and stability.
In fact it is known that with the increasing amount of renewables, the grid is being more exposed to fluctation 
and stochasticty mainly driven by the weather forecast, requiring the energy storage and ancillary service includes 
managing voltage levels and frequency to prevent grid disturbances.
Another part of the literature is focusing on scheduling and resource allocation under uncertainty. 
Methods like stochastic or robust optimization are now becoming more popular in order to minimize Value at Risk (VAr), under uncertain conditions. 

\subsection{Cloud Stuff }

Power electronic converters (PECs) are ubiquitous and play a crucial role in renewable sources integration, serving 
as interconnection between the loads, energy storage systems and the grid.
Numerous are the application of power converters 
for example in Electrical Vehicles (EVs) converting alternating current (AC) to direct current (DC) or in 
photovoltaic panels converting DC to AC, to match the grid frequency and voltage. 
Moreover, PECs are crucial in the operation of industrial electric drives to meet 
motors requirements in high power application. 
For instance, three-phase converters are commonly used in industrial settings 
due to their ability to handle high power requirements efficiently.
It is clear that PECs are could play a major role in orchestrating loads, 
by leveraging their flexibility and adaptability. 
However, the aggregation of this flexibilities come with a price,
especially in terms of data aggregation, computational resources, scalability and advances control 
techniques that are able to interact with cloud architecture, PLCs and APIs. (Read paper Musumeci) 


Small summaries of paper analyzing could platform with gree transitions


\cite{bagherzadeh2020integration} analyzes the challanges in communication, storages and computational capabilities in 
massive streams of data, while \cite{shahinzadeh2018green} presents how IoT benefit the accurate forecasting and predictive mantainance ensuring high security levels. 
Paper \cite{hossein2020internet} extensively review the literature on the application of IoT in energy sectors ans smart grids, by 
distinguishes the the transmission and distribution levels, where IoT can be applied to energy efficiency, aggregation of distributed generations and electric vehicles
aggregation (V2G), from the demand side where IoT can be used for battery energy storage management and control to smart building control. 

Wi-Fi, Bluetooth, ZigBee \cite{karunarathne2018wireless} LTE-4G and 5G networks \cite{li20185g}