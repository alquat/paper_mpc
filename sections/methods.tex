\section{Proposed Methodology}
\subsection{ARX Model}
	We have chosen to represent the system in \textit{discrete time},
	
	\begin{align}
		y_t &+ a_1 y_{t-1} + \cdots + a_n y_{t-p} = b_1 u_{t-1} + \cdots + b_q u_{t-q}
	\end{align}

	\begin{align}
		\theta = \begin{bmatrix}
		a_1, \ldots, a_p, b_1, \ldots, b_q
		\end{bmatrix}^T
	\end{align} 

	\begin{align}
		\varphi(t) = \begin{bmatrix}
		-y_{t-1} \cdots - y_{t-p}, u_{t-1} \cdots u_{t-q}
		\end{bmatrix}^T
	\end{align}

	\begin{align}
		y_t = \varphi^T_t \theta_t \quad  \forall t \in (1, N)
	\end{align}



	For a given system we can collect inputs and outputs over a time
	interval $t$

	\begin{align}
		\emph{Z}^N = \big \{u(t_0), y(t_0), \cdots, u(N), y(n)\big \}
	\end{align}


	\begin{align}
		\hat{\theta} = \arg\min_{x} \emph{V}_N(\theta) + \lambda(\theta - \theta^*)R(\theta - \theta^*)
	\end{align}

	The ARX(p, q) model is given by:
	\begin{align}
	y_t = \sum_{i=1}^p \varphi_i y_{t-i} + \sum_{j=1}^q \beta_j x_{t-j} + \epsilon_t
	\end{align}



\subsection{Inflow Estimation}

	The considered wastewater pumping station, only the outflow and the height are measured, 
	while the inflow is not measured or estimated. Therefore, in order to estimate the inflow of the tank, 
	we implemented an observer by using the tank equation where the instantaneous rate of change of the height $dh$ is proportional 
	to the difference between the inflow and the outflow of the tank, scaled by the area of the tank, as pointed Eq.\ref{eq:height_eq}.
	\begin{align}
	h_t = h_0 + \frac{T_s}{A}(Q_{in,t} - Q_{out,t}) + v_t \label{eq:height_eq}
	\end{align}

	\noindent where \( A \), the cross-sectional area of the tank, $T_s$ is the sampling time of the measurements,
	$Q_{in,t}$ and $Q_{out,t}$ are the measured inflow and outflow respectively, and $v_t$ is the measurement noise. 

	By defining the the inflow  \( Q_{in,t} \) any time \( t \), the hidden state the 
	Eq.\ref{eq:height_eq}, \( w_t \) the process noise, \( Q \) is the covariance of the process noise, and \( R \) is the measurement noise covariance.
	\[
	Q_{in, t+1} = Q_{in, t} + w_t
	\]
	
	\noindent The states can be estimated by means of the Kalman filter, at any given time \( t \)
	\subsubsection*{Prediction Step}
	\begin{align*}
	\hat{Q}_{in, t+1|t} &= \hat{Q}_{in,t|t} \\
	P_{t+1|t} &= P_{t|t} + Q
	\end{align*}
	where \( Q \) is the covariance of the process noise.

	\subsubsection*{Update Step}
	Compute the Kalman Gain and update the estimate with the measurement:

	\begin{align*}
		\Delta h_{t+1} &= h_{t+1} - z_{h,t+1} \\
		\Delta Q_{t+1} &= \hat{Q}_{in,t+1|t} - z_{Q_{out},t+1} \\
		\hat{Q}_{in,t+1|t+1} &= \hat{Q}_{in,t+1|t} + K_{t+1} \left(\frac{A}{T_s}\Delta h_{t+1} - \Delta Q_{t+1} \right)
		\end{align*}
	
\


\subsection{Inflow Forecast}
The inflow estimation is the a first step to enhable the control of the system from a lower
perspective. This means that the recursive estimation of the inflows allows, as it will be clarified in the next 
steps, to maintain the equilibrium between the inflow and the incoming outflow. This estimation can be performed 
at different time level, in order to accommodate the system. An example is that if the system is running at 
seconds resolution, then the estimation can support the balancing equation of the height. However, after visual 
inspection, it can be seen that the inflow is mainly driven by two large scale phenomena, i.e., the human 
behavior and the rain. In fact, in our application the inflow peaks in the morning, registering the 
most intensive water usage during the day. Additionally, the inflow patter follows a 24 hours seasonality with trend 
following the alternation of meteorological seasons. However, on the top of this components largely characterized by the 
large scale human behavior, the inflow is strongly affected by the rain. In fact, even in separate systems, the rain 
infiltration can strongly affect the inflow and thus the normal capacity operations of pumping or treatment stations, representing the main cause of 
overflow in most of the swage operations. Mitigation of overflow is among the challenges in modern pumping and treatment stations, but still represent 
one of the main driver of clean water pollution. Therefore, preventing this uncontrolled scenarios plays a pivotal economical, environmental and social issue. 
Nevertheless, building accurate inflow forecast is challenging for many reason, i.e., uncertainty in the rain forecast, mismatch between the 
rainfall forecasted and the actual precipitation and finally the availability of the historical rain forecasts. 
For this purpose, we build our own data collection, where the rainfall forecats are queried hourly from the Norwegian Meteorological Institute (Meteorologisk institutt), while the 
actual precipitation is retrieved from the Danish Meteorological Institute (Danmarks Meteorologiske Institut), which provides, on a per minute basis,  measurements of 
the precipitation occurred in the past minute. Finally, the measured rainfall is resampled hourly, and use to train the model, while the forecast 
are used as future covariates to provide hourly forecast over an horizon of 1 day.
Given the decomposable structure of the time series and the impulsive behavior caused by the rain, we found
an online hybrid windowed model be the best fit for this application. The hybrid strategy can be devided into
two different steps. The first one if time series modeling using Generalized Additing Models (GAM), where the time 
series is modeled as follows: 

\begin{equation}
	y_t = g_t + s_t + r_t + \epsilon_t
\end{equation}

where: 
\begin{itemize}
	\item[-] $\boldsymbol{g_t}$ represents the trend, modeled as piecewise linear trend with changes points:  
\end{itemize}
\begin{equation}
	g_t = (k+a_t^T \delta)t + (m + a_t^T\gamma)
\end{equation}
with k defined as the growing rate, $\delta$ the rate adjustments, m an offset parameter, and $\gamma$
a term to ensure the continuity of the function. 
\begin{itemize}
	\item[-] $\boldsymbol{s_t}$ is the seasonality terms given by the N first Fourier terms, as reported in Eq.\ref{gam:seas} 
\end{itemize}

\begin{equation}
	\label{gam:seas}
	s_t = \sum_{n=1}^{N} \Bigg (a_n cos \Bigg(\frac{2\pi nt}{P}\Bigg) + b_n sin \Bigg(\frac{2\pi nt}{P}\Bigg)\Bigg)
\end{equation}

\begin{itemize}
	\item[-] whereas, $\boldsymbol{r_t}$ is the external covariates representing the rain forecasted for the next 24 hours. 
\end{itemize}

Once fitted the model, assuming the residuals are almost normally distributed and thta the time 
series is stationary, we model the residuals of the \( \epsilon_{t, GAM} \) of a time series \( Q_{in,t} \), using an ARIMA(\(p, d, q\)) model:

\begin{equation}
	\Phi(B)(1 - B)^d \epsilon_{, GAM} = \Theta(B) \nu_t
\end{equation}

to fit these model we used the Pophet model developed by facebook using a recursive fitting widown 
and a simple arima for the remaning autocorrelation of the residuals. 

\subsection{Hierarchical Model Predictive Control}
	\subsubsection{Higher-Level MPC}


\begin{subequations}\label{P0:min_max}
	\begin{align}
		\underset{\eta}{\text{min}} & \underset{Q_{in,t} \in \mathcal{U}_t}{\text{max}} \quad \sum_{t=1}^{T} \left( \eta_t \mathcal{C}_t + \Lambda_t^T \sigma_t \right) \label{P0:min_max_1} \\
		\text{s.t.} \quad & \sigma_t \geq 0, \ell \geq 0  \quad \forall t \in T \label{P0:min_max_2} \\
		& \eta_{t} = \sum_{j=1}^3 \eta_{j,t-1}, \quad \forall t \label{P0:min_max_3} \\
		& h_{min} \leq h_{ref, t} \leq h_{max}, \quad \forall t \label{P0:min_max_5} \\
		& E_{min} \leq E_{t} \leq E_{max}, \quad \forall t \label{P0:min_max_6} \\
		& \hat{Q}_{in,t} = \hat{Q}_{in,t-n} \label{P0:min_max_7}
	\end{align}
	\end{subequations}

\subsubsection{Lower-Level MPC}
	\begin{subequations}\label{P2:FinalModel}
		\begin{align}
			\underset{\omega, E, P, Q_{\text{out}}}{\text{min}}& \sum_{k=1}^{h} \mathcal{Q} \left\lVert h_k - h_{ref} \right\rVert^2 + \mathcal{R} \left\lVert \omega_k \right\rVert^2 + \Gamma \left\lVert E \right\rVert^{2}_{k} + \Lambda_k^T \sigma_k  \label{P1:lower_mpc} \\
			&\hspace{-1em} \text{s.t.}  \quad \quad \forall \sigma_k \geq 0, \; \ell \geq 0 \quad \forall k \in h \label{P1:1} \\
			& \qquad Q_{\text{out}, k} = \sum_{j=1}^3 Q_{out_{j,k-1}} \label{P1:2} \\
			& \qquad E_{k} = \sum_{j=1}^3 E_{k-\ell} \label{P1:3}  \\
			& \qquad P_{k} = \sum_{j=1}^3 P_{k-\ell} \label{P1:4}  \\
			& \qquad Q_{in,k} = \tilde{Q}_{in,k-1} \label{P1:5}  \\
			& \qquad h_k = \frac{1}{A} \Big(\tilde{Q}_{in,k} - Q_{out,k}\Big) \label{P1:6} \\
			& \qquad P_{min} - \sigma_{P,t} \leq P_k \leq P_{max} + \sigma_{P,t} \label{P1:8} \\
			& \qquad \omega_{min} - \sigma_{\omega,t} \leq \omega_k \leq \omega_{max} + \sigma_{\omega,t} \label{P1:7} \\
			& \qquad h_{ref} - \sigma_{h,t} \leq h_k \leq h_{ref} + \sigma_{h,t} \label{P1:9}
	\end{align}
	\end{subequations}



\RestyleAlgo{ruled}
\SetKw{KwSet}{Set}
\begin{algorithm}
	\KwSet $h \leftarrow$ lower-MPC horizon\\
	\KwSet $H \leftarrow$ upper-MPC horizon\\
	\KwSet $t \leftarrow 0$ \\
	\caption{Hierarchical MPC of aggregation of power converters.}\label{alg:two}
	\While(){control = True}{
		$t \leftarrow t+1$ \\
		$k \leftarrow k+1$ \\
		$\hat{q}_{out,k}\leftarrow$  measure outflow state \\
		$\hat{h}_{k}\leftarrow$  measure height state \\
		$\hat{q}_{in, k|k} \leftarrow$ estimate inflow state \\
		
		\If{$k \bmod H = 0$}{
			$r_{t+H} \leftarrow$ update rain forecast from API \\
			$\pi_{t+H} \leftarrow$ update price forecast from API \\
			$\tilde{Q}_{in, t+H|k} \leftarrow$ forecast inflow state\\
			$h_{ref} \leftarrow$ \textbf{solve upper-OCP (scheduling)}
			}
		%update inflow forecast 
	
	$\{\omega_{k}\}_{j \in {1,2,3}}$ \textbf{solve lower-OCP (drives control)} \\ 
	apply $\{\omega_{k}\}_{j \in {1,2,3}}$ to the system
	}
	

\end{algorithm}