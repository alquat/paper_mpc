\section{System Identification}
\subsection{ARX Model}
	We have chosen to represent the system in \textit{discrete time},
	\begin{align}
		y_t &+ a_1 y_{t-1} + \cdots + a_n y_{t-p} = b_1 u_{t-1} + \cdots + b_q u_{t-q}
	\end{align}

	\begin{align}
		\theta = \begin{bmatrix}
		a_1, \ldots, a_p, b_1, \ldots, b_q
		\end{bmatrix}^T
	\end{align} 

	\begin{align}
		\varphi(t) = \begin{bmatrix}
		-y_{t-1} \cdots - y_{t-p}, u_{t-1} \cdots u_{t-q}
		\end{bmatrix}^T
	\end{align}

	\begin{align}
		y_t = \varphi^T_t \theta_t \quad  \forall t \in (1, N)
	\end{align}



	For a given system we can collect inputs and outputs over a time
	interval $t$

	\begin{align}
		\emph{Z}^N = \big \{u(t_0), y(t_0), \cdots, u(N), y(n)\big \}
	\end{align}


	\begin{align}
		\hat{\theta} = \arg\min_{x} \emph{V}_N(\theta) + \lambda(\theta - \theta^*)R(\theta - \theta^*)
	\end{align}

	The ARX(p, q) model is given by:
	\begin{align}
	y_t = \sum_{i=1}^p \varphi_i y_{t-i} + \sum_{j=1}^q \beta_j x_{t-j} + \epsilon_t
	\end{align}


\subsection{Piecewise Linear function}

	\begin{algorithm}
		\caption{Grouping Time Series Data by Bin Medians}
		\begin{algorithmic}[1]
		\REQUIRE Time series data \( y = \{y_1, y_2, \ldots, y_T\} \), excitation data \( u = \{u_1, u_2, \ldots, u_T\} \), bin size \( n \)
		\ENSURE Grouped data by bin medians
		
		\STATE Divide the time series \( y \) into bins of size \( n \)
		\FOR{$i = 1$ to $\left\lceil \frac{T}{n} \right\rceil$}
			\STATE $B_i \gets \{ y_{(i-1)n+1}, y_{(i-1)n+2}, \ldots, y_{\min(in, T)} \}$
			\STATE Compute the median \( m_i \) of the bin \( B_i \)
		\ENDFOR
		
		\STATE Group the data in \( y \) by their corresponding bin medians \( m_i \)
		
		\end{algorithmic}
		\end{algorithm}


\section{Inflow Estimation}

	The considered wastewater pumping station, only the outflow and the height are measured, 
	while the inflow is not measured or estimated. Therefore, in order to estimate the inflow of the tank, 
	we implemented an observer by using the tank equation where the instantaneous rate of change of the height $dh$ is proportional 
	to the difference between the inflow and the outflow of the tank, scaled by the area of the tank, as pointed Eq.\ref{eq:height_eq}.
	\begin{align}
	h_t = h_0 + \frac{T_s}{A}(Q_{in,t} - Q_{out,t}) + v_t \label{eq:height_eq}
	\end{align}

	\noindent where \( A \), the cross-sectional area of the tank, $T_s$ is the sampling time of the measurements,
	$Q_{in,t}$ and $Q_{out,t}$ are the measured inflow and outflow respectively, and $v_t$ is the measurement noise. 

	By defining the the inflow  \( Q_{in,t} \) any time \( t \), the hidden state the 
	Eq.\ref{eq:height_eq}, \( w_t \) the process noise, \( Q \) is the covariance of the process noise, and \( R \) is the measurement noise covariance.
	\[
	Q_{in, t+1} = Q_{in, t} + w_t
	\]
	
	\noindent The states can be estimated by means of the Kalman filter, at any given time \( t \)
	\subsubsection*{Prediction Step}
	\begin{align*}
	\hat{Q}_{in, t+1|t} &= \hat{Q}_{in,t|t} \\
	P_{t+1|t} &= P_{t|t} + Q
	\end{align*}
	where \( Q \) is the covariance of the process noise.

	\subsubsection*{Update Step}
	Compute the Kalman Gain and update the estimate with the measurement:

	\begin{align*}
		\Delta h_{t+1} &= h_{t+1} - z_{h,t+1} \\
		\Delta Q_{t+1} &= \hat{Q}_{in,t+1|t} - z_{Q_{out},t+1} \\
		\hat{Q}_{in,t+1|t+1} &= \hat{Q}_{in,t+1|t} + K_{t+1} \left(\frac{A}{T_s}\Delta h_{t+1} - \Delta Q_{t+1} \right)
		\end{align*}
	
\


\section{Inflow Forecast}

\section{Hierarchical Model Predictive Control}
	\subsection{Higher-Level MPC}
		\begin{subequations}\label{P0:higher_mpc}
			\begin{align}
				\underset{\zeta, }{\text{min}}& \sum_{t=1}^{T} \Gamma \left\lVert E \right\rVert^{2}_{t} + \Lambda_t^T \sigma_t   \label{P0:1} \\
				&\hspace{-1em} \text{s.t.}  \quad \quad \forall \sigma_t \geq 0 \label{P0:2} \\
				& \qquad \zeta_{t} = \sum_{j=1}^3 \zeta_{t-1} \label{P0:3}  \\
				& \qquad Q_{in,t} \leq 0\label{P0:4} \\
				& \qquad h_{min}\leq h_{t} \leq h_{max}\label{P0:5}\\
				& \qquad E_{min}\leq E_{t} \leq E_{max}\label{P0:6}
		\end{align}
		\end{subequations}
	

\subsection{Lower-Level MPC}
	\begin{subequations}\label{P2:FinalModel}
		\begin{align}
			\underset{\omega, E, P, Q_{\text{out}}}{\text{min}}& \sum_{k=1}^{h} \mathcal{Q} \left\lVert h_k - h_r \right\rVert^2 + \mathcal{R} \left\lVert \omega_k \right\rVert^2 + \Gamma \left\lVert E \right\rVert^{2}_{k} + \Lambda_k^T \sigma_k  \label{P1:lower_mpc} \\
			&\hspace{-1em} \text{s.t.}  \quad \quad \forall \sigma_k \geq 0, \; l \geq 0 \label{P1:1} \\
			& \qquad Q_{\text{out}, k} = \sum_{j=1}^3 Q_{out_{j,k-1}} \label{P1:2} \\
			& \qquad E_{k} = \sum_{j=1}^3 E_{k-l} \label{P1:3}  \\
			& \qquad P_{k} = \sum_{j=1}^3 P_{k-l} \label{P1:4}  \\
			& \qquad Q_{in,k} = \tilde{Q}_{in,k-1} \label{P1:5}  \\
			& \qquad h_k = \frac{1}{A} \Big(\tilde{Q}_{in,k} - Q_{out,k}\Big) \label{P1:6} \\
			& \qquad \omega_{l} - \sigma_{\omega} \leq \omega_k \leq \omega_{u} + \sigma_{\omega} \label{P1:7} \\
			& \qquad P_{l} - \sigma_{P} \leq P_k \leq P_{u} + \sigma_{P} \label{P1:8} \\
			& \qquad h_{r} - \sigma_{h_{r}} \leq h_k \leq h_{l} + \sigma_{h_{r}} \label{P1:9}
	\end{align}
	\end{subequations}